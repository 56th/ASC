%%%%%%%%%%%%%%%%%%%%%%%%%%%%%%%%%%%%%%%%%
% Jacobs Landscape Poster
% LaTeX Template
% Version 1.0 (29/03/13)
%
% Created by:
% Computational Physics and Biophysics Group, Jacobs University
% https://teamwork.jacobs-university.de:8443/confluence/display/CoPandBiG/LaTeX+Poster
% 
% Further modified by:
% Nathaniel Johnston (nathaniel@njohnston.ca)
%
% This template has been downloaded from:
% http://www.LaTeXTemplates.com
%
% License:
% CC BY-NC-SA 3.0 (http://creativecommons.org/licenses/by-nc-sa/3.0/)
%
%%%%%%%%%%%%%%%%%%%%%%%%%%%%%%%%%%%%%%%%%

%----------------------------------------------------------------------------------------
%	PACKAGES AND OTHER DOCUMENT CONFIGURATIONS
%----------------------------------------------------------------------------------------

\documentclass[final, svgnames]{beamer}

\usepackage[scale=1.24]{beamerposter} % Use the beamerposter package for laying out the poster

% \usetheme{confposter} % Use the confposter theme supplied with this template

\setbeamercolor{block title}{fg=Blue,bg=white} % Colors of the block titles
\setbeamercolor{block body}{fg=black,bg=white} % Colors of the body of blocks
\setbeamercolor{block alerted title}{fg=white,bg=dblue!70} % Colors of the highlighted block titles
\setbeamercolor{block alerted body}{fg=black,bg=dblue!10} % Colors of the body of highlighted blocks
% Many more colors are available for use in beamerthemeconfposter.sty

%-----------------------------------------------------------
% Define the column widths and overall poster size
% To set effective sepwid, onecolwid and twocolwid values, first choose how many columns you want and how much separation you want between columns
% In this template, the separation width chosen is 0.024 of the paper width and a 4-column layout
% onecolwid should therefore be (1-(# of columns+1)*sepwid)/# of columns e.g. (1-(4+1)*0.024)/4 = 0.22
% Set twocolwid to be (2*onecolwid)+sepwid = 0.464
% Set threecolwid to be (3*onecolwid)+2*sepwid = 0.708

\newlength{\sepwid}
\newlength{\onecolwid}
\newlength{\twocolwid}
\newlength{\threecolwid}
\setlength{\paperwidth}{48in} % A0 width: 46.8in
\setlength{\paperheight}{36in} % A0 height: 33.1in
\setlength{\sepwid}{0.015\paperwidth} % Separation width (white space) between columns
\setlength{\onecolwid}{0.2\paperwidth} % Width of one column
\setlength{\twocolwid}{0.464\paperwidth} % Width of two columns
\setlength{\threecolwid}{0.708\paperwidth} % Width of three columns
\setlength{\topmargin}{-0.5in} % Reduce the top margin size
%-----------------------------------------------------------

\usepackage{graphicx}  % Required for including images

\usepackage{booktabs} % Top and bottom rules for tables

%----------------------------------------------------------------------------------------
%	TITLE SECTION 
%----------------------------------------------------------------------------------------

\title[ASC($n$) Method]{Generalized  Approximate  Static  Condensation  Method  for  a  Heterogeneous  Multi-material Diffusion Problem}
\author[Alexander Zhiliakov]{
	\small{\underline{Alexander Zhiliakov}\inst{1}, Daniil Svyatsky\inst{2}, Maxim Olshanskii\inst{1}, Eugene Kikinzon\inst{2}, Mikhail Shashkov\inst{2}}
	\vskip -1mm
}
\institute[UH, LANL] {
	\begin{tabular}[.8]{c c c}
		\inst{1}Department of Mathematics & \qquad & \inst{2}Los Alamos \\
		University of Houston & & National Laboratory \\
		\includegraphicsw[.2]{logo_uh.png} & & \includegraphicsw[.2]{logo_lanl.png}
	\end{tabular}
	\vskip 3mm
	\tiny{This work was performed under the auspices of the US Department of Energy at Los Alamos National Laboratory under contract DE-AC52-06NA25396; LA-UR-19-20919} 
}

%----------------------------------------------------------------------------------------

% bold for everything
\usepackage{bm}
% lowercase mathcal font
\usepackage{dutchcal}
\hypersetup{
	colorlinks,
	allcolors=.,
	urlcolor=blue,
	filecolor=blue
}
% braces for subeqns and boxes
\usepackage{empheq}
% http://mirror.hmc.edu/ctan/macros/latex/contrib/mathtools/empheq.pdf
\newcommand*\widefbox[1]{\fbox{\hspace{1em}#1\hspace{1em}}}
% hl
\usepackage{soul}
\makeatletter
\let\HL\hl
\renewcommand\hl{%
	\let\set@color\beamerorig@set@color
	\let\reset@color\beamerorig@reset@color
	\HL}
\makeatother
% sub figures / grids of pictures
\usepackage{subcaption} 
\graphicspath{{../img/}} % includegraphics path
\newcommand{\includegraphicsw}[2][1.]{\includegraphics[width=#1\linewidth]{#2}}
\newcommand{\svginputw}[2][\linewidth]{\def\svgwidth{#1}\input{../img/#2}} % pdf_tex path
% tables
\let\oldtabular\tabular
\renewcommand{\tabular}[1][1.5]{\def\arraystretch{#1}\oldtabular}
\usepackage{hhline}
\usepackage{multirow}
% \coloneqq
\usepackage{mathtools}
% math commands for convinience
\DeclareMathOperator{\argmin}{arg\,min}
% bold vectors
\newcommand{\vect}[1]{\boldsymbol{\mathbf{#1}}}

\newcommand{\bcell}{T}
\newcommand{\bmesh}{{\vect{\mathcal T}}}
\newcommand{\mmesh}{{\vect{\mathcal \tau}}}
\newcommand{\bfaces}[1][]{{\vect{\mathcal F}_{\text{#1}}}}
\newcommand{\mfaces}[1][]{{\vect{\mathcal f}_{\text{#1}}}}

\newcommand{\LTwo}{{\mathbb L^2}}
\newcommand{\lTwo}{{\mathcal l^2}}
\newcommand{\HDiv}{{\mathbb H_\text{div}}}
\newcommand{\Rn}[1]{{\mathbb R^{#1}}}
\newcommand{\Pn}[1]{{\mathbb P^{#1}}}
\newcommand{\LTwoSpace}[1][\Omega]{{\mathbb L^2\left({#1}\right)}}
\newcommand{\HSpace}[1]{{\mathbb H^{#1}\left(\Omega\right)}}
\newcommand{\lTwoSpace}[1][\Omega]{{\mathcal l^2\left({#1}\right)}}
\newcommand{\HDivSpace}[1][\Omega]{{\mathbb H_\text{div}\left({#1}\right)}}
\newcommand{\PnSpace}[2]{{\mathbb P^{#1}\left({#2}\right)}}

\newcommand{\errlTwo}[1]{e^{\lTwo}_{#1}}
\newcommand{\errInf}[1]{e^{\infty}_{#1}}

% differentials
\newcommand*\diff{\mathop{}\!\mathrm{d}}
\newcommand*\Diff[1]{\mathop{}\!\mathrm{d^#1}}

\DeclareMathOperator{\Ker}{Ker}

\newcommand{\Ltwo}{\mathbb L^2}
\newcommand{\LSpace}[1][\Omega]{\mathbb L^2\left({#1}\right)}

\newenvironment{braced}
{\par\smallskip\hbox to\columnwidth\bgroup
	\hss$\left\{\begin{minipage}{\columnwidth}}
{\end{minipage}\right.$\hss\egroup\smallskip}

% https://tex.stackexchange.com/a/160827/135296
\newcommand\Wider[2][3em]{%
	\makebox[\linewidth][c]{%
		\begin{minipage}{\dimexpr\textwidth+#1\relax}
			\raggedright#2
		\end{minipage}%
	}%
}

\setbeamercolor{block body example}{bg=red!20!white}
\setbeamercolor{block title example}{fg=red, bg=red!40!white}

\begin{document}

\addtobeamertemplate{block end}{}{\vspace*{2ex}} % White space under blocks
\addtobeamertemplate{block alerted end}{}{\vspace*{2ex}} % White space under highlighted (alert) blocks

\setlength{\belowcaptionskip}{2ex} % White space under figures
\setlength\belowdisplayshortskip{2ex} % White space under equations

\begin{frame}[t] % The whole poster is enclosed in one beamer frame

\begin{columns}[t] % The whole poster consists of three major columns, the second of which is split into two columns twice - the [t] option aligns each column's content to the top

\begin{column}{\sepwid}\end{column} % Empty spacer column

\begin{column}{\onecolwid} % The first column

%----------------------------------------------------------------------------------------
%	OBJECTIVES
%----------------------------------------------------------------------------------------

%----------------------------------------------------------------------------------------
%	INTRODUCTION
%----------------------------------------------------------------------------------------

{\Large \textcolor{Blue}{Generalized  Approximate  Static  Condensation  Method  for  a  Heterogeneous  Multi-material Diffusion Problem}}
\vskip .45cm
\underline{Alexander Zhiliakov} (alex@math.uh.edu), Daniil Svyatsky, Maxim Olshanskii, Eugene Kikinzon, Mikhail Shashkov
\vskip .45cm 

\begin{block}{Problem Setting}

Our objective is to solve the diffusion problem in the mixed form
\begin{empheq}[left = \empheqlbrace]{alignat=2}
\vect K^{-1}\,\vect u &+ \nabla\,p\,&= 0&\quad\text{in } \Omega \subset \Rn{2}, \nonumber \\
\nabla\cdot\vect u    &+ c\,p       &= f&\quad\text{in } \Omega, \nonumber
\end{empheq}
with boundary data
\begin{equation*}
p = g_D \quad\text{on } \partial\Omega_D, \quad
\vect u \cdot \hat{\vect n} = g_N \quad\text{on } \partial\Omega_N.
\end{equation*}
\textbf{Challenges}:
\begin{itemize}
	\item The diffusion tensor~$\vect K$ may sharply vary in~$\Omega$ and may be discontinuous
	\item We want to use general polygonal meshes, and
	\item \textcolor{Red}{be able to handle material interfaces not aligned with the mesh}  
\end{itemize}

\begin{figure}
	\centering
	\begin{subfigure}{.33\linewidth}
		\centering
		\includegraphicsw{ring_base_voronoi.png}
		\caption{Macro-mesh~$\bmesh$}
	\end{subfigure}%
	\hfill
	\begin{subfigure}{.33\linewidth}
		\centering
		\includegraphicsw{ring_mmcs_voronoi.png}
		\caption{Multi-mat. cells}
	\end{subfigure}%
	\hfill
	\begin{subfigure}{.33\linewidth}
		\centering
		\includegraphicsw{ring_mini_voronoi.png}
		\caption{MOF}
	\end{subfigure}
\end{figure}
Moment-of-fluid interface reconstruction $\Rightarrow$ reconstructed interface may be discontinuous


\begin{columns}
	\begin{column}{.4\textwidth}
		\begin{flushright}
			Consider $T \in \bmesh_H$:
		\end{flushright}
	\end{column}%
	\begin{column}{.35\textwidth}
		\begin{empheq}[left = \empheqlbrace]{alignat=2}
		\vect K^{-1}\,\vect u &+ \nabla\,p\,    &= 0       &\quad\text{in } T, \nonumber \\
		\nabla\cdot\vect u    &+ c\,p           &= f       &\quad\text{in } T, \nonumber \\
		&\phantom{+cc\,}p &= \textcolor{Red}{\lambda} &\quad\text{on } \partial T \nonumber
		\end{empheq}
	\end{column}
	\begin{column}{.25\textwidth}
	\end{column}
\end{columns}
$$\Downarrow$$
		\vskip -3cm
		\begin{figure}
			\begin{subfigure}{.3\linewidth}\centering
				\includegraphicsw[.9]{ring_mini_voronoi_cell.png}
				Minimesh~$\mmesh_h$ of $T$
			\end{subfigure}%
			\begin{subfigure}{.1\linewidth}\centering
				\vskip -1.65cm
				$\Rightarrow\:\,$
			\end{subfigure}%
			\begin{subfigure}{.6\linewidth}\centering
				\begin{exampleblock}{Discretization:}\centering
					Apply Mimetic Finite Difference\\Method\textsuperscript{*}
				\end{exampleblock}
				$\Downarrow$
				\begin{equation*}
				\begin{pmatrix}
				\phantom{-}\vect M_\mmesh & \vect B^T_\mmesh \\
				-\vect B_\mmesh & \vect \Sigma_\mmesh            
				\end{pmatrix} 
				\begin{pmatrix}
				\vect u_\mmesh \\
				\vect p_\mmesh        
				\end{pmatrix}
				= 
				\begin{pmatrix}
				\vect E_\mmesh\,\vect C_\mmesh\,\textcolor{Red}{\vect \lambda_\mmesh} \\
				\vect f_\mmesh        
				\end{pmatrix}
				\end{equation*}		
			\end{subfigure}%
		\end{figure}
		\setul{1pt}{.4pt} % https://tex.stackexchange.com/questions/50637/closer-underline
		\tiny{
			\textsuperscript{*}L.\,Beirao da Veiga, K.\,Lipnikov, G.\,Manzini\\
			$\:$\href{https://www.springer.com/us/book/9783319026626}{\ul{The Mimetic Finite Difference Method for Elliptic Problems}}\\
			$\:$Springer 2014
		}
\end{block}

\begin{block}{Description of the Method}

		If one knows the \textcolor{Red}{pressure trace~$\lambda$} for each~$T \in \bmesh$, one can recover the solution in~$\bmesh$. The idea is \textbf{(i)} to express external flux DOFs in terms of \textcolor{Red}{trace DOFs} (\textit{static condensation}),
$$
\vect u^\text{ext}_\mmesh \coloneqq \vect E_\mmesh^T\,\vect u_\mmesh = \vect A_\mmesh\,\vect C_\mmesh\,\textcolor{Red}{\vect \lambda_\mmesh} - \vect a_\mmesh,
$$
and \textbf{(ii)} to get the system for \textcolor{Red}{trace DOFs} by requiring weak continuity of fluxes. \textbf{Problem}: we may have different number of \textcolor{Red}{trace DOFs} from~$T^+$ and~$T^-$.	
%\begin{figure}
%	\centering
%	%\caption{MMCs: $3 = \#\,\mfaces^+_F \ne \#\,\mfaces^-_F = 2$, $\nu_F = 3$}
%	\svginputw[.6\linewidth]{e_plus_e_minus.pdf_tex}
%\end{figure}
		\textbf{Solution}: approximate a pressure trace on~$F$ with a polynomial~$\lambda_F \in \PnSpace{n}{F}$ described in terms of its~$(n+1)$ moments
\begin{equation*}
\lambda^{(i)}_F=\frac{\int_F  \lambda\,s_i \diff l}{|F|}, \quad i = 0, \dots, n.
%\frac{\int_F \hat \lambda\,s_i \diff l}{|F|}, \quad i = 0, \dots, n.
\end{equation*}
Here~$s_i \in \PnSpace{i}{F}$ is a fixed polynomial of degree~$i$ such that~$s_i \perp_\LTwo s_j$, $j < i$
\end{block}


%----------------------------------------------------------------------------------------

\end{column} % End of the first column

\begin{column}{\sepwid}\end{column} % Empty spacer column

\begin{column}{\onecolwid} % Begin (column 2)



%----------------------------------------------------------------------------------------
%	METHODS
%----------------------------------------------------------------------------------------

\begin{block}{Description of the Method (cont.)}
\begin{figure}
	\begin{subfigure}{.05\textwidth}
		\rotatebox{90}{\textcolor{Red}{DOFs}}
	\end{subfigure}%
	\begin{subfigure}{.95\textwidth}
		\begin{braced}
			Now we express trace DOFs on mini\,faces of~$\mmesh$ via coarse trace DOFs~$\coloneqq (n+1)$~moments on each base\,face of~$T$,
			\begin{align*}
			\vect \lambda_\mmesh &= \vect R_\mmesh\,\vect \lambda_T \quad \Rightarrow \\
			\vect u^\text{ext}_\mmesh &= \vect A_\mmesh\,\vect C_\mmesh\,\vect R_\mmesh\,\vect \lambda_T - \vect a_\mmesh,
			\end{align*}  
		\end{braced}
	\end{subfigure}
	\begin{subfigure}{.05\textwidth}
		\rotatebox{90}{\textcolor{Red}{Constraints}}
	\end{subfigure}%
	\begin{subfigure}{.95\textwidth}
		\begin{braced}
			and close the system by requiring weak continuity of normal fluxes on each base face
			\begin{equation*}
			\int_F \vect u|_{T^+}\cdot\hat{\vect n}\,s_i \diff l = \int_F \vect u|_{T^-}\cdot\hat{\vect n}\,s_i \diff l, \: i = 0, \dots, n, F \in \bfaces[int]
			\end{equation*}	
		\end{braced}
	\end{subfigure}
\end{figure}
Express fluxes in terms of traces $\Rightarrow$ get SLAE for coarse trace DOFs

\begin{equation*}
\int_F \vect u|_{T^+}\cdot\hat{\vect n}\,s_i \diff l = \int_F \vect u|_{T^-}\cdot\hat{\vect n}\,s_i \diff l, \: i = 0, \dots, n \text{ for } F \in \bfaces[int]
\end{equation*}
$$\Downarrow$$
\begin{align*}
n = 0: \quad& \sum_{f\in \mfaces_F(T^+)} u^{\text{ext}}_{\mmesh^+}(f)\,|f| + \sum_{f\in \mfaces_F(T^-)} u^{\text{ext}}_{\mmesh^-}(f)\,|f| = 0, \\
n = 1: \quad& \sum_{f\in \mfaces_F(T^+)} u^{\text{ext}}_{\mmesh^+}(f)\,\int_{f} s_1 \diff{l} + \sum_{f\in \mfaces_F(T^-)} u^{\text{ext}}_{\mmesh^-}(f)\,\int_{f} s_1 \diff{l} = 0
\end{align*}
$$\Downarrow$$
\begin{equation*}
\left( \vect R^T_{\mmesh^+}\,\vect C_{\mmesh^+}\,\vect u^{\text{ext}}_{\mmesh^+} \right)_{i+m} + \left( \vect R^T_{\mmesh^-}\,\vect C_{\mmesh^-}\,\vect u^{\text{ext}}_{\mmesh^-} \right)_{j+m} = 0, \quad m \in \{0, 1\}
\end{equation*}
$$\Downarrow$$
\begin{align*}\small
		\begin{split}
		\Big( \underbrace{\left( \vect R^T_{\mmesh^+}\,\vect C_{\mmesh^+}\,\vect A_{\mmesh^+}\,\vect C_{\mmesh^+}\,\vect R_{\mmesh^+} \right)}_{\vect S_{T^+} \coloneqq} \vect \lambda_{T^+} \Big)_{i+m} 
		&+ 
		\Big( \underbrace{\left( \vect R^T_{\mmesh^-}\,\vect C_{\mmesh^-}\,\vect A_{\mmesh^-}\,\vect C_{\mmesh^-}\,\vect R_{\mmesh^-} \right)}_{\vect S_{T^-} \coloneqq} \vect \lambda_{T^-} \Big)_{j+m} = \\
		\big( \underbrace{\vect R^T_{\mmesh^+}\,\vect C_{\mmesh^+}\,\vect a_{\mmesh^+}}_{\vect s_{T^+}} \big)_{i+m}
		&+
		\big( \underbrace{\vect R^T_{\mmesh^-}\,\vect C_{\mmesh^-}\,\vect a_{\mmesh^-}}_{\vect s_{T^-}} \big)_{j+m}
		\end{split}
\end{align*}
$$\Downarrow$$
\begin{empheq}[box=\widefbox]{align*}
\vect S_\bmesh &= \sum_{T \in \bmesh} \vect N^T_T\,\vect S_T\,\vect N_T, & \normalsize{\text{Global system:}} \\
\vect s_\bmesh &= \sum_{T \in \bmesh} \vect N^T_T\,\vect s_T, & \normalsize{\textcolor{Red}{\vect S_\bmesh\,\vect \lambda_\bmesh = \vect s_\bmesh}}
\end{empheq}\normalsize
\begin{itemize}
	\item \textbf{Theorem}: system matrix~$\vect S_\bmesh$ is sparse and SPD for ASC(0) and ASC(1)
	\item Hence efficient solvers and preconditioners are available (e.\,g. CG + Algebraic Multigrid)
	\item Once we obtain~$\vect \lambda_\bmesh$, we recover pressure and flux DOFs in each cell~$T \in \bmesh$ (this may be done in parallel) 
\end{itemize}
\end{block}


\begin{block}{Numerical Experiments}

\textbf{Algebraic Robustness}. We solve the diffusion problem w/ $\vect K = k\,\vect I$, $k = 1$ on the left part and .1 on the right. Exact solution is piecewise linear
\begin{figure}
	\centering
	\caption{$w \coloneqq$ width of the left minimesh cells}
	\begin{subfigure}{.33\linewidth}
		\centering
		\includegraphicsw{skew1.png}
		\caption{$w = .1$}
	\end{subfigure}%
	\hfill
	\begin{subfigure}{.33\linewidth}
		\centering
		\includegraphicsw{skew01.png}
		\caption{$w = .01$}
	\end{subfigure}%
	\hfill
	\begin{subfigure}{.33\linewidth}
		\centering
		\includegraphicsw{skew001.png}
		\caption{$w = .001$}
	\end{subfigure}
\end{figure}
$\kappa_{\text{ASC(0)}}$ does not depend on $w$, and $\kappa_{\text{ASC(1)}}$ is proportional to $w^{-1}$. However, if we remove 3 smallest eig values (corresponding to 3 int MM faces),

\end{block}


\end{column} % End of the second column




\begin{column}{\onecolwid} % The third column

%----------------------------------------------------------------------------------------
%	CONCLUSION
%----------------------------------------------------------------------------------------

\begin{block}{Numerical Experiments (cont.)}

\textbf{we will have $\tilde{\vect\kappa}_{\text{ASC(1)}} \approx \vect\kappa_{\text{ASC(0)}}$}.
Starting from some iteration CG behaves like extreme eig values are not present; that is, several small eig values is not a problem
\begin{figure}
	\centering
	\caption{Condition Numbers of ASC(0)\,/\,ASC(1) Matrices} 
	\begin{subfigure}{.45\linewidth}
		\centering\tiny
		\begin{tabular}[1.2]{ | c | c | c | c |}
			\hline
			$w$ & $\kappa_{\text{ASC(0)}}$ & $\kappa_{\text{ASC(1)}}$ & $\widetilde\kappa_{\text{ASC(1)}}$\\
			\hline
			$10^{-1}$ & 41.0 & 1\,730       & 41.0  \\
			\hline
			$10^{-2}$ & 45.2 & 2\,817       & 45.1  \\
			\hline
			$10^{-3}$ & 48.3 & 16\,391      & 48.3   \\
			\hline
			$10^{-4}$ & 49.0 & 152\,325     & 49.0    \\
			\hline
			$10^{-5}$ & 49.1 & $1.5\times10^6$&49.1  \\
			\hline
		\end{tabular}%
		%			\begin{tabular}[1.2]{ | c | c | c | }
		%				\hline
		%				$w$ & $\kappa_{\text{ASC(0)}}$ & $\kappa_{\text{ASC(1)}}$ \\
		%				\hline
		%				$10^{-1}$ & 41 & 1\,730 \\ 
		%				\hline
		%				$10^{-2}$ & 45 & 2\,817 \\
		%				\hline
		%				$10^{-3}$ & 48 & 16\,391 \\
		%				\hline
		%				$10^{-4}$ & 49 & 152\,325 \\
		%				\hline
		%				$10^{-5}$ & 49 & $1.5\times10^6$ \\
		%				\hline
		%			\end{tabular}
	\end{subfigure}%
	\hfill
	\begin{subfigure}{.55\linewidth}
		\centering
		\includegraphicsw{logplot.png}
	\end{subfigure}
\end{figure}
If the base mesh consists of triangles + we have no material interfaces, ASC($n$) boils down to Mixed-Hybrid Raviart\,--\,Thomas FEM:
\begin{align*}
\|\vect u - \vect u_h\|_{\LTwoSpace} &\le c\,h\,\|\vect u\|_{\HSpace{1}},\\
\|p - p_h\|_{\LTwoSpace} &\le c \left( h\,\|p\|_{\HSpace{1}} + h^2\,\|p\|_{\HSpace{2}} \right).
\end{align*}
That is, we cannot expect ASC($n$) convergence to be better than linear. We define \textbf{discrete $\LTwo$-norm}
\begin{equation*}
\| v \|_{\lTwoSpace} \coloneqq \| P_h\,v \|_{\LTwoSpace} \le \| v \|_{\LTwoSpace},
\end{equation*} 
where $P_h \coloneqq$ $\LTwo$-projection operator on the space of piecewise constant functions on each cell~$T \in \bmesh$ (or on each~$\tau \in \mmesh$ if $T$ is a MMC)
\vskip 1cm
\textbf{ASC(0) $\rightarrow$ ASC(1): Motivation}. If $\vect K_i \equiv \vect K_j$ and the exact soln is linear, ASC(0) produces errors due to const trace approximation, and ACS(1) recovers the exact soln
\vskip 1cm
\textbf{Piecewise $P_1$ Solution}. We solve the diffusion problem on the sequence of square meshes w/ $\vect K = k\,\vect I$, $k = 1$ on the left part and .1 on the right. Exact solution is piecewise linear
\begin{figure}
	\centering
	\begin{subfigure}{.45\linewidth}
		\centering
		\includegraphicsw{skew_ref.png}
		\caption{Exact soln, $p$}
	\end{subfigure}%
	\hfill
	\begin{subfigure}{.45\linewidth}
		\centering
		\includegraphicsw{skew_geometry_square.png}
		\caption{Materials}
	\end{subfigure}
\end{figure}
{\centering\small
\begin{tabular}[1.2]{| c | c || c | c | c || c | c |}
	\hline
	\multirow{5}{*}{\rotatebox{90}{ASC(0)}} & $h$ & $\errlTwo{p}$ & $\rho_p$ & $\errInf{p}$ & $\errlTwo{u}$ & $\rho_u$ \\
	\cline{2-7}
	& $3.5\times10^{-1}$ & $7.3\times10^{-1}$ &     & 4.8               & $ 6.6\times10^{-1}$ &  \\
	\cline{2-7}
	& $8.8\times10^{-2}$ & $1.6\times10^{-1}$ & 1.1 & 1.2               & $ 3.5\times10^{-1}$ & 0.46 \\ 
	\cline{2-7}
	& $2.2\times10^{-2}$ & $3.7\times10^{-2}$ & 1.1 & $3.4\times10^{-1}$ & $ 1.3\times10^{-1}$ & 0.71 \\ 
	\cline{2-7}
	& $5.5\times10^{-3}$ & $8.9\times10^{-3}$ & 1.0 & $7.9\times10^{-2}$ & $ 4.1\times10^{-2}$ & 0.83 \\
	\hline
	\hline
	\multirow{5}{*}{\rotatebox{90}{ASC(1)}} & $h$ & $\errlTwo{p}$ & $\rho_p$ & $\errInf{p}$ & $\errlTwo{u}$ & $\rho_u$ \\
	\cline{2-7}
	& $3.5\times10^{-1}$ & $2.5\times10^{-2}$ & & $2.9\times10^{-1}$      & $ 4.6\times10^{-2}$ &   \\
	\cline{2-7}
	& $8.8\times10^{-2}$ & $1.9\times10^{-3}$ & 1.84 & $6.6\times10^{-2}$ & $ 2.0\times10^{-2}$ & 0.6 \\
	\cline{2-7}
	& $2.2\times10^{-2}$ & $1.6\times10^{-4}$ & 1.79 & $4.3\times10^{-2}$ & $ 5.5\times10^{-3}$ & 0.93 \\
	\cline{2-7}
	& $5.5\times10^{-3}$ & $1.3\times10^{-5}$ & 1.80 & $2.0\times10^{-2}$ & $ 1.3\times10^{-3}$ & 1.   \\
	\hline
\end{tabular}}
\vskip 1cm
\textbf{Piecewise $P_2$ Solution w/ 3 Materials}. We solve the diffusion problem on triangular meshes w/ $\vect K = k\,\vect I$, $k = 1$ outside the ring and .001 inside. Exact solution is piecewise quadratic
\begin{figure}
	\centering
	\begin{subfigure}{.45\linewidth}
		\centering
		\includegraphicsw{ring_ref_mesh.png}
		\caption{Exact soln, $p$}
	\end{subfigure}%
	\hfill
	\begin{subfigure}{.45\linewidth}
		\centering
		\includegraphicsw{ring_ref_slice.png}
		\caption{$p(x,\frac{1}{2})$}
	\end{subfigure}
\end{figure}

\end{block}

\end{column} % End of the third column


\begin{column}{\sepwid}\end{column} % Empty spacer column

\begin{column}{\onecolwid} % Begin (column 2)

\begin{block}{Numerical Experiments (cont.)}

{\footnotesize
\begin{table}
	\begin{subtable}{.6\linewidth}
		\begin{tabular}[1.1]{| c | c || c | c || c |}
			\hline
			\multirow{7}{*}{\rotatebox{90}{ASC(0)}} & $h$ & $\errlTwo{p}$ & $\rho_p$ & $\errInf{p}$ \\
			\cline{2-5}
			& $3.0\times10^{-1}$ & 4.5 & & 17 \\
			\cline{2-5}
			& $2.5\times10^{-1}$ & 4.5 & & 17 \\
			\cline{2-5}
			& $1.3\times10^{-1}$ & 4.0 & & 17 \\
			\cline{2-5}
			& $8.3\times10^{-2}$ & 4.4 & & 17 \\
			\cline{2-5}
			& \hl{$6.7\times10^{-2}$} & $7.1\times10^{-1}$ & & 4.9 \\
			\cline{2-5}
			& $4.3\times10^{-2}$ & $4.5\times10^{-1}$ & 1.2 & 5.0 \\
			\hline
			\hline
			\multirow{7}{*}{\rotatebox{90}{ASC(1)}} & $h$ & $\errlTwo{p}$ & $\rho_p$ & $\errInf{p}$ \\
			\cline{2-5}
			& $3.0\times10^{-1}$ & $4.5\times10^{-1}$ & & 3.5 \\
			\cline{2-5}
			& $2.5\times10^{-1}$ & $2.6\times10^{-1}$ & 3 & 2.7 \\
			\cline{2-5}
			& $1.3\times10^{-1}$ & $9.2\times10^{-2}$ & 1.5 & $6.2\times10^{-1}$ \\
			\cline{2-5}
			& $8.3\times10^{-2}$ & $4.8\times10^{-2}$ & 1.6 & $8.3\times10^{-1}$ \\
			\cline{2-5}
			& $6.7\times10^{-2}$ & $2.8\times10^{-2}$ & 2.5 & $2.3\times10^{-1}$ \\
			\cline{2-5}
			& $4.3\times10^{-2}$ & $1.0\times10^{-2}$ & 2.3 & $6.3\times10^{-2}$ \\
			\hline
		\end{tabular}%
	\end{subtable}%
	\begin{subtable}{.4\linewidth}
		\begin{tabular}[1.1]{| c | c || c | c || c |}
			\hline
			\multirow{7}{*}{\rotatebox{90}{Arithmetic homo.}} & $h$ & $\errlTwo{\text{p}}$ & $\rho_p$ & $\errInf{p}$ \\
			\cline{2-5}
			& $3.0\times10^{-1}$ & 4.9 & & 17 \\
			\cline{2-5}
			& $2.5\times10^{-1}$ & 5.0 & & 17 \\
			\cline{2-5}
			& $1.3\times10^{-1}$ & 4.9 & & 17 \\
			\cline{2-5}
			& $8.3\times10^{-2}$ & 4.7 & & 17 \\
			\cline{2-5}
			& {$6.7\times10^{-2}$} & 4.4 & & 16 \\
			\cline{2-5}
			& $4.3\times10^{-2}$ & $9.7\times10^{-1}$ & 3.5 & 5.7 \\
			\hline
			\hline
			\multirow{7}{*}{\rotatebox{90}{Harmonic homo.}} & $h$ & $\errlTwo{p}$ & $\rho_p$ & $\errInf{p}$ \\
			\cline{2-5}
			& $3.0\times10^{-1}$ & 2.3 & & 15 \\
			\cline{2-5}
			& $2.5\times10^{-1}$ & 1.7 & 1.6 & 16 \\
			\cline{2-5}
			& $1.3\times10^{-1}$ & $7.3\times10^{-1}$ & 1.2 & 12 \\
			\cline{2-5}
			& $8.3\times10^{-2}$ & $4.8\times10^{-1}$ & 1.0 & 12 \\
			\cline{2-5}
			& $6.7\times10^{-2}$ & $3.4\times10^{-1}$ & 1.6 & 9.4 \\
			\cline{2-5}
			& $4.3\times10^{-2}$ & $1.6\times10^{-1}$ & 1.7 & 8.2 \\
			\hline
		\end{tabular}
	\end{subtable}
\end{table}}
\vskip .45cm
Before \hl{$h = 6.7\times10^{-2}$} we have cells\,/\,faces with 3 materials, and after this mesh level we have only 2 material MMCs
\vskip 1cm
\textbf{Unsteady Problem}. We solve the unsteady diffusion problem on Voronoi meshes w/ $\vect K = k\,\vect I$, $k = .002$ inside the ring, 10 in the inner disk, and 1 elsewhere. We set~$g_D = 1$ on the left bndry and 0 on the right; top and bottom bndries are insulated. Equilibrium state is $p \equiv 1$
\begin{figure}
	\centering
	\begin{subfigure}{.4\linewidth}
		\centering
		\includegraphicsw{transient2/supermesh.png}
		\caption{Conforming mesh}
	\end{subfigure}%
	\hfill
	\begin{subfigure}{.58\linewidth}
		\centering
		\includegraphicsw{transient2/ref_slices.png}
		\caption{Cuts~$p_*\big((x,0.5), t\big)$ of the ref soln}	
	\end{subfigure}
\end{figure}
\begin{figure}
	\centering
	\begin{subfigure}{.25\linewidth}
		\centering
		\includegraphicsw{transient2/movie_frames/ref/frame_1s34.png}
		\caption{\tiny Reference}
	\end{subfigure}%
	\hfill
	\begin{subfigure}{.25\linewidth}
		\centering
		\includegraphicsw{transient2-1/movie_frames/arithmetic_homo/frame_1s43.png}
		\caption{\tiny Arithmetic homo.}	
	\end{subfigure}%
	\hfill
	\begin{subfigure}{.25\linewidth}
		\centering
		\includegraphicsw{transient2-1/movie_frames/harmonic_homo/frame_1s43.png}
		\caption{\tiny Harmonic homo.}	
	\end{subfigure}%
	\par
	\begin{subfigure}{.25\linewidth}
		\centering
		\includegraphicsw{transient2-1/movie_frames/asc0/frame_1s43.png}
		\caption{\tiny ASC(0)}
	\end{subfigure}%
	\hfill
	\begin{subfigure}{.25\linewidth}
		\centering
		\includegraphicsw{transient2-1/movie_frames/asc1/frame_1s43.png}
		\caption{\tiny ASC(1)}	
	\end{subfigure}%
	\hfill
	\begin{subfigure}{.25\linewidth}
		\centering
		\includegraphicsw{transient2-1/movie_frames/asc_diff/frame_1s43.png}
		\caption{\tiny ASC(0,\,1) difference}	
	\end{subfigure}%
\end{figure}\centering
{\small Comparison of the discrete solutions~$p_h$, $h = 1.5 \times 10^{-1}$, $t = 1.25$}

\end{block}

\begin{block}{Summary}

\begin{itemize}
	\item ASC($n$) is able to efficiently handle unfitted material interfaces 
	\item 2\textsuperscript{nd} order $\lTwo$-convergence for ASC(1) 
	\item Effective condition number seems to be uniformly bounded w.r.t. an interface position
	\item The underline matrix is SPD and sparse; its pattern does not depend on mini\,meshes
	\item A.\,Zhiliakov et al. \href{https://www.researchgate.net/publication/330912268_A_higher_order_approximate_static_condensation_method_for_multi-material_diffusion_problems}{\ul{A higher order approximate static condensation method for multi-material diffusion problems}}, 2019 (JCP preprint)
	\item \textbf{TODO List}: Anisotropic diffusion: homogenization is not applicable; what about ASC($n$)? Extension to 3D? 
\end{itemize}
\vskip .45cm 
\small This work was performed under the auspices of the US Department of Energy at Los Alamos National Laboratory under contract DE-AC52-06NA25396; LA-UR-19-20919
\end{block}

\end{column} % End of the second column



\end{columns} % End of all the columns in the poster
\end{frame} % End of the enclosing frame

\end{document}

