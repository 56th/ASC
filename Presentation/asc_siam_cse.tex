\documentclass[svgnames]{beamer} % for xcolor https://latex.org/forum/viewtopic.php?t=2445 
\mode<presentation>
{
\usetheme{Warsaw}
\setbeamertemplate{page number in head/foot}[totalframenumber]
%\setbeamertemplate{footline}[frame number]
\setbeamertemplate{headline}{}
\setbeamercovered{transparent}
% or whatever (possibly just delete it)
}

\usepackage[english]{babel}
\usepackage[utf8]{inputenc}

\title[ASC($n$) Method]{A higher order approximate static condensation method for multi-material diffusion problems}
\author[Alexander Zhiliakov]{
	\small{\underline{Alexander Zhiliakov}\inst{1}, Daniil Svyatsky\inst{2}, Maxim Olshanskii\inst{1}, Eugene Kikinzon\inst{2}, Mikhail Shashkov\inst{2}}
	\vskip -1mm
}
\institute[UH, LANL] {
	\begin{tabular}[.8]{c c c}
		\inst{1}Department of Mathematics & \qquad & \inst{2}Los Alamos \\
		University of Houston & & National Laboratory \\
		\includegraphicsw[.2]{logo_uh.png} & & \includegraphicsw[.2]{logo_lanl.png}
	\end{tabular}
	\vskip 3mm
	\tiny{This work was performed under the auspices of the US Department of Energy at Los Alamos National Laboratory under contract DE-AC52-06NA25396; LA-UR-19-20919} 
}
\date[Oct 5\,--\,7, 2018]{\small\href{https://www.siam.org/Conferences/CM/Main/cse19}{\underline{SIAM CSE}}, 27 Feb 2019}

\usepackage{media9} % https://tex.stackexchange.com/questions/240243/getting-gif-and-or-moving-images-into-a-latex-presentation
% bold for everything
\usepackage{bm}
% lowercase mathcal font
\usepackage{dutchcal}
\hypersetup{
	colorlinks,
	allcolors=.,
	urlcolor=blue,
	filecolor=blue
}
% braces for subeqns and boxes
\usepackage{empheq}
% http://mirror.hmc.edu/ctan/macros/latex/contrib/mathtools/empheq.pdf
\newcommand*\widefbox[1]{\fbox{\hspace{1em}#1\hspace{1em}}}
% hl
\usepackage{soul}
\makeatletter
\let\HL\hl
\renewcommand\hl{%
	\let\set@color\beamerorig@set@color
	\let\reset@color\beamerorig@reset@color
	\HL}
\makeatother
% sub figures / grids of pictures
\usepackage{subcaption} 
\graphicspath{{../img/}} % includegraphics path
\newcommand{\includegraphicsw}[2][1.]{\includegraphics[width=#1\linewidth]{#2}}
\newcommand{\svginputw}[2][\linewidth]{\def\svgwidth{#1}\input{../img/#2}} % pdf_tex path
% tables
\let\oldtabular\tabular
\renewcommand{\tabular}[1][1.5]{\def\arraystretch{#1}\oldtabular}
\usepackage{hhline}
\usepackage{multirow}
% \coloneqq
\usepackage{mathtools}
% math commands for convinience
\DeclareMathOperator{\argmin}{arg\,min}
% bold vectors
\newcommand{\vect}[1]{\boldsymbol{\mathbf{#1}}}

\newcommand{\bcell}{T}
\newcommand{\bmesh}{{\vect{\mathcal T}}}
\newcommand{\mmesh}{{\vect{\mathcal \tau}}}
\newcommand{\bfaces}[1][]{{\vect{\mathcal F}_{\text{#1}}}}
\newcommand{\mfaces}[1][]{{\vect{\mathcal f}_{\text{#1}}}}

\newcommand{\LTwo}{{\mathbb L^2}}
\newcommand{\lTwo}{{\mathcal l^2}}
\newcommand{\HDiv}{{\mathbb H_\text{div}}}
\newcommand{\Rn}[1]{{\mathbb R^{#1}}}
\newcommand{\Pn}[1]{{\mathbb P^{#1}}}
\newcommand{\LTwoSpace}[1][\Omega]{{\mathbb L^2\left({#1}\right)}}
\newcommand{\HSpace}[1]{{\mathbb H^{#1}\left(\Omega\right)}}
\newcommand{\lTwoSpace}[1][\Omega]{{\mathcal l^2\left({#1}\right)}}
\newcommand{\HDivSpace}[1][\Omega]{{\mathbb H_\text{div}\left({#1}\right)}}
\newcommand{\PnSpace}[2]{{\mathbb P^{#1}\left({#2}\right)}}

\newcommand{\errlTwo}[1]{e^{\lTwo}_{#1}}
\newcommand{\errInf}[1]{e^{\infty}_{#1}}

% differentials
\newcommand*\diff{\mathop{}\!\mathrm{d}}
\newcommand*\Diff[1]{\mathop{}\!\mathrm{d^#1}}

\DeclareMathOperator{\Ker}{Ker}

\newcommand{\Ltwo}{\mathbb L^2}
\newcommand{\LSpace}[1][\Omega]{\mathbb L^2\left({#1}\right)}

\newenvironment{braced}
{\par\smallskip\hbox to\columnwidth\bgroup
	\hss$\left\{\begin{minipage}{\columnwidth}}
{\end{minipage}\right.$\hss\egroup\smallskip}

% https://tex.stackexchange.com/a/160827/135296
\newcommand\Wider[2][3em]{%
	\makebox[\linewidth][c]{%
		\begin{minipage}{\dimexpr\textwidth+#1\relax}
			\raggedright#2
		\end{minipage}%
	}%
}


\begin{document}

	\begin{frame}
		\titlepage
	\end{frame}

	\begin{frame}{Overview}
		\tableofcontents
	\end{frame}

	\section{The ASC($n$) Method}
	
	\subsection{Problem Setting}
	
	\begin{frame}{Diffusion Problem}
		Our objective is to solve the diffusion problem in the mixed form
		\begin{empheq}[left = \empheqlbrace]{alignat=2}
			\vect K^{-1}\,\vect u &+ \nabla\,p\,&= 0&\quad\text{in } \Omega \subset \Rn{2}, \nonumber \\
			\nabla\cdot\vect u    &+ c\,p       &= f&\quad\text{in } \Omega, \nonumber
		\end{empheq}
		with boundary data
		\begin{align*}
			p &= g_D \quad\text{on } \partial\Omega_D, \\
			\vect u \cdot \hat{\vect n} &= g_N \quad\text{on } \partial\Omega_N.
		\end{align*}
		\textbf{Challenges}:
		\begin{itemize}
			\item The diffusion tensor~$\vect K$ may sharply vary in~$\Omega$ and may be discontinuous
			\item We want to use general polygonal meshes, and
			\item \textcolor{Red}{be able to handle material interfaces not aligned with the mesh}  
		\end{itemize}
	\end{frame}

	\begin{frame}{Mesh}
		\begin{figure}
			\centering
			\begin{subfigure}{.33\linewidth}
				\centering
				\only<1>{\includegraphicsw{ring_base_voronoi.png}}\only<2>{\includegraphicsw{ring_base_voronoi_refined.png}}
				\caption{Macro-mesh~$\bmesh$}
			\end{subfigure}%
			\hfill
			\begin{subfigure}{.33\linewidth}
				\centering
				\only<1>{\includegraphicsw{ring_mmcs_voronoi.png}}\only<2>{\includegraphicsw{ring_mmcs_voronoi_refined.png}}
				\caption{Multi-material cells}
			\end{subfigure}%
			\hfill
			\begin{subfigure}{.33\linewidth}
				\centering
				\only<1>{\includegraphicsw{ring_mini_voronoi.png}}\only<2>{\includegraphicsw{ring_mini_voronoi_refined.png}}
				\caption{MOF}
			\end{subfigure}
		\end{figure}
		Moment-of-fluid interface reconstruction $\Rightarrow$ reconstructed interface may be discontinuous
	\end{frame}

	\begin{frame}{Local Problem}
		\begin{columns}
			\begin{column}{.4\textwidth}
				\begin{flushright}
					Consider $T \in \bmesh_H$:
				\end{flushright}
			\end{column}%
			\begin{column}{.35\textwidth}
				\begin{empheq}[left = \empheqlbrace]{alignat=2}
					\vect K^{-1}\,\vect u &+ \nabla\,p\,    &= 0       &\quad\text{in } T, \nonumber \\
					\nabla\cdot\vect u    &+ c\,p           &= f       &\quad\text{in } T, \nonumber \\
					&\phantom{+cc\,}p &= \textcolor{Red}{\lambda} &\quad\text{on } \partial T \nonumber
				\end{empheq}
			\end{column}
			\begin{column}{.25\textwidth}
			\end{column}
		\end{columns}
		$$\Downarrow$$
		\begin{overlayarea}{\textwidth}{5cm}	
			\only<1>{%
				\vskip -.5cm
				\begin{empheq}[box=\widefbox]{equation}
					\begin{split}
						&\text{Find trial functions $\langle \vect u, p \rangle \in \HDivSpace[T] \times \LTwoSpace[T]$ such that} \\
						&\left\{
						\begin{aligned} 
						\int_T \vect K^{-1}\,\vect u\cdot\vect v \diff \vect x - \int_T p\,\nabla\cdot\vect v \diff \vect x &= -\int_{\partial T} \textcolor{Red}{\lambda}\,\vect v\cdot\hat{\vect n} \diff l, \\
						\int_T \nabla\cdot\vect u\,q \diff \vect x + \int_T c\,p\,q \diff \vect x &= \int_T f\,q \diff \vect x
						\end{aligned} 
						\right. \\ 
						&\text{holds for all test functions $\langle \vect v, q \rangle \in \HDivSpace[T] \times \LTwoSpace[T]$} \nonumber
					\end{split}
				\end{empheq}
			}%
			\only<2>{%
			\vskip -.5cm
			\begin{figure}
				\begin{subfigure}{.3\linewidth}\centering
					\includegraphicsw[.9]{ring_mini_voronoi_cell.png}
					Minimesh~$\mmesh_h$ of $T$
				\end{subfigure}%
				\begin{subfigure}{.1\linewidth}\centering
					\vskip -1.65cm
					$\Rightarrow\:\,$
				\end{subfigure}%
				\begin{subfigure}{.6\linewidth}\centering
					\begin{block}{Discretization}\centering
						Apply Mimetic Finite Difference\\Method\textsuperscript{*}
					\end{block}
					$\Downarrow$
					\begin{equation*}
						\begin{pmatrix}
							\phantom{-}\vect M_\mmesh & \vect B^T_\mmesh \\
							-\vect B_\mmesh & \vect \Sigma_\mmesh            
							\end{pmatrix} 
							\begin{pmatrix}
							\vect u_\mmesh \\
							\vect p_\mmesh        
						\end{pmatrix}
						= 
						\begin{pmatrix}
							\vect E_\mmesh\,\vect C_\mmesh\,\textcolor{Red}{\vect \lambda_\mmesh} \\
							\vect f_\mmesh        
						\end{pmatrix}
					\end{equation*}		
				\end{subfigure}%
			\end{figure}
			\setul{1pt}{.4pt} % https://tex.stackexchange.com/questions/50637/closer-underline
			\tiny{
				\textsuperscript{*}L.\,Beirao da Veiga, K.\,Lipnikov, G.\,Manzini\\
				$\:$\href{https://www.springer.com/us/book/9783319026626}{\ul{The Mimetic Finite Difference Method for Elliptic Problems}}\\
				$\:$Springer 2014
			}
		}
		\end{overlayarea}	
	\end{frame}

	\subsection{Description of the Method}
	
	\begin{frame}{Approximate \ul{Static Condensation}}
		If one knows the \textcolor{Red}{pressure trace~$\lambda$} for each~$T \in \bmesh$, one can recover the solution in~$\bmesh$. The idea is \textbf{(i)} to express external flux DOFs in terms of \textcolor{Red}{trace DOFs} (\textit{static condensation}),
		$$
			\vect u^\text{ext}_\mmesh \coloneqq \vect E_\mmesh^T\,\vect u_\mmesh = \vect A_\mmesh\,\vect C_\mmesh\,\textcolor{Red}{\vect \lambda_\mmesh} - \vect a_\mmesh,
		$$
		and \textbf{(ii)} to get the system for \textcolor{Red}{trace DOFs} by requiring weak continuity of fluxes. 
%		\begin{equation*}
%			\int_F \vect u|_{T^+}\cdot\hat{\vect n}\,s_i \diff l = \int_F \vect u|_{T^-}\cdot\hat{\vect n}\,s_i \diff l, \quad i = 0, \dots, n \quad \text{for each } F \in \bfaces[int]
%		\end{equation*}
		\textbf{Problem}: we may have different number of \textcolor{Red}{trace DOFs} from~$T^+$ and~$T^-$	
		\begin{figure}
			\centering
			%\caption{MMCs: $3 = \#\,\mfaces^+_F \ne \#\,\mfaces^-_F = 2$, $\nu_F = 3$}
			\svginputw[.6\linewidth]{e_plus_e_minus.pdf_tex}
		\end{figure}
	\end{frame}

	\begin{frame}{\ul{Approximate} Static Condensation}
		\begin{figure}
			\centering
			%\caption{MMCs: $3 = \#\,\mfaces^+_F \ne \#\,\mfaces^-_F = 2$, $\nu_F = 3$}
			\svginputw[.6\linewidth]{e_plus_e_minus.pdf_tex}
		\end{figure}
		\textbf{Solution}: approximate a pressure trace on~$F$ with a polynomial~$\lambda_F \in \PnSpace{n}{F}$ described in terms of its~$(n+1)$ moments
		\begin{equation*}
			\lambda^{(i)}_F=\frac{\int_F  \lambda\,s_i \diff l}{|F|}, \quad i = 0, \dots, n.
			%\frac{\int_F \hat \lambda\,s_i \diff l}{|F|}, \quad i = 0, \dots, n.
		\end{equation*}
		Here~$s_i \in \PnSpace{i}{F}$ is a fixed polynomial of degree~$i$ such that~$s_i \perp_\LTwo s_j$, $j < i$
	\end{frame}

	\begin{frame}{ASC($n$): DOFs and Constraints}
		\begin{figure}
			\begin{subfigure}{.05\textwidth}
				\rotatebox{90}{\textcolor{Red}{DOFs}}
			\end{subfigure}%
			\begin{subfigure}{.95\textwidth}
				\begin{braced}
					Now we express trace DOFs on mini\,faces of~$\mmesh$ via coarse trace DOFs~$\coloneqq (n+1)$~moments on each base\,face of~$T$,
					\begin{align*}
					\vect \lambda_\mmesh &= \vect R_\mmesh\,\vect \lambda_T \quad \Rightarrow \\
					\vect u^\text{ext}_\mmesh &= \vect A_\mmesh\,\vect C_\mmesh\,\vect R_\mmesh\,\vect \lambda_T - \vect a_\mmesh,
					\end{align*}  
				\end{braced}
			\end{subfigure}
			\begin{subfigure}{.05\textwidth}
				\rotatebox{90}{\textcolor{Red}{Constraints}}
			\end{subfigure}%
			\begin{subfigure}{.95\textwidth}
				\begin{braced}
					and close the system by requiring weak continuity of normal fluxes on each base face
					\begin{equation*}
					\int_F \vect u|_{T^+}\cdot\hat{\vect n}\,s_i \diff l = \int_F \vect u|_{T^-}\cdot\hat{\vect n}\,s_i \diff l, \: i = 0, \dots, n \text{ for } F \in \bfaces[int]
					\end{equation*}	
				\end{braced}
			\end{subfigure}
		\end{figure}
		Express fluxes in terms of traces $\Rightarrow$ get SLAE for coarse trace DOFs
	\end{frame}

	\subsection{ASC(0) and ASC(1)}

	\begin{frame}{ASC(1) DOFs}
		\vskip -.5cm
		\begin{figure}\tiny
			\begin{subfigure}{.45\linewidth}
				\centering
				\svginputw{e_plus_1.pdf_tex}
			\end{subfigure}%
			\hfill
			\begin{subfigure}{.45\linewidth}
				\centering
				\svginputw{e_plus_2.pdf_tex}
			\end{subfigure}
		\end{figure}
		\vskip -.5cm
		\begin{columns}
			\begin{column}{.5\textwidth}
				\begingroup
				\setlength\arraycolsep{1pt}
				\begin{equation*}\tiny
					\underbrace{
						\begin{pmatrix}
							\lambda_{11}(\mmesh) \\
							\lambda_{12}(\mmesh) \\
							\lambda_{13}(\mmesh) \\
							\lambda_{21}(\mmesh) \\
							\lambda_{22}(\mmesh) \\
							\lambda_{23}(\mmesh) \\
							\lambda_{31}(\mmesh) \\
							\lambda_{32}(\mmesh) \\
							\lambda_{41}(\mmesh)
						\end{pmatrix}
					}_{\vect \lambda_\mmesh =}
					= 
					\underbrace{\begin{pmatrix}
						1 & s_{11} & 0 & 0 & 0 & 0 & 0 \\
						1 & s_{12} & 0 & 0 & 0 & 0 & 0 \\
						1 & s_{13} & 0 & 0 & 0 & 0 & 0 \\
						0 & 0 & 1 & s_{21} & 0 & 0 & 0 \\
						0 & 0 & 1 & s_{22} & 0 & 0 & 0 \\
						0 & 0 & 1 & s_{23} & 0 & 0 & 0 \\
						0 & 0 & 0 & 0 & 1 & s_{31} & 0 \\
						0 & 0 & 0 & 0 & 1 & s_{32} & 0 \\
						0 & 0 & 0 & 0 & 0 & 0 & 1 \\
						\end{pmatrix}}_{\vect R_\mmesh =}
					\underbrace{\begin{pmatrix}
						\lambda^{(0)}_1(T) \\
						\lambda^{(1)}_1(T) \\
						\lambda^{(0)}_2(T) \\
						\lambda^{(1)}_2(T) \\
						\lambda^{(0)}_3(T) \\
						\lambda^{(1)}_3(T) \\
						\lambda^{(0)}_4(T)
					\end{pmatrix}}_{\vect \lambda_T =}
				\end{equation*}
				\endgroup
			\end{column}
			\begin{column}{.5\textwidth}
				Here $s_{ij} \coloneqq$ signed distance between $j$th mini-face and $i$th macro-face centroids. 
				For ASC(0) the matrix $\vect R_\mmesh$  is as on the left with rows containing $s_{ij}$-s eliminated.
			\end{column}
		\end{columns}
	\end{frame}

	\begin{frame}{ASC(1) System Assembly}\footnotesize
		\begin{equation*}
			\int_F \vect u|_{T^+}\cdot\hat{\vect n}\,s_i \diff l = \int_F \vect u|_{T^-}\cdot\hat{\vect n}\,s_i \diff l, \: i = 0, \dots, n \text{ for } F \in \bfaces[int]
		\end{equation*}
		$$\Downarrow$$
		\begin{align*}
			n = 0: \quad& \sum_{f\in \mfaces_F(T^+)} u^{\text{ext}}_{\mmesh^+}(f)\,|f| + \sum_{f\in \mfaces_F(T^-)} u^{\text{ext}}_{\mmesh^-}(f)\,|f| = 0, \\
			n = 1: \quad& \sum_{f\in \mfaces_F(T^+)} u^{\text{ext}}_{\mmesh^+}(f)\,\int_{f} s_1 \diff{l} + \sum_{f\in \mfaces_F(T^-)} u^{\text{ext}}_{\mmesh^-}(f)\,\int_{f} s_1 \diff{l} = 0
		\end{align*}
%		\begin{align*}
%			n = 0: \quad& \sum u^{\text{ext}}_{\mmesh^+\,i}\,|\mfaces^+_{F\,i}| + \sum u^{\text{ext}}_{\mmesh^-\,i}\,|\mfaces^-_{F\,i}| = 0, \\
%			n = 1: \quad& \sum u^{\text{ext}}_{\mmesh^+\,i}\,\Delta s^+_i\,|\mfaces^+_{F\,i}| + \sum u^{\text{ext}}_{\mmesh^-\,i}\,\Delta s^-_i\,|\mfaces^-_{F\,i}| = 0
%		\end{align*}
		$$\Downarrow$$
		\begin{equation*}
			\left( \vect R^T_{\mmesh^+}\,\vect C_{\mmesh^+}\,\vect u^{\text{ext}}_{\mmesh^+} \right)_{i+m} + \left( \vect R^T_{\mmesh^-}\,\vect C_{\mmesh^-}\,\vect u^{\text{ext}}_{\mmesh^-} \right)_{j+m} = 0, \quad m \in \{0, 1\}
		\end{equation*}
		$$\Downarrow$$
		\begin{overlayarea}{\textwidth}{3cm}
		\only<1>{%
			\vskip -.5cm
			\begin{align*}
				\begin{split}
				\Big( \underbrace{\left( \vect R^T_{\mmesh^+}\,\vect C_{\mmesh^+}\,\vect A_{\mmesh^+}\,\vect C_{\mmesh^+}\,\vect R_{\mmesh^+} \right)}_{\vect S_{T^+} \coloneqq} \vect \lambda_{T^+} \Big)_{i+m} 
				&+ 
				\Big( \underbrace{\left( \vect R^T_{\mmesh^-}\,\vect C_{\mmesh^-}\,\vect A_{\mmesh^-}\,\vect C_{\mmesh^-}\,\vect R_{\mmesh^-} \right)}_{\vect S_{T^-} \coloneqq} \vect \lambda_{T^-} \Big)_{j+m} = \\
				\big( \underbrace{\vect R^T_{\mmesh^+}\,\vect C_{\mmesh^+}\,\vect a_{\mmesh^+}}_{\vect s_{T^+}} \big)_{i+m}
				&+
				\big( \underbrace{\vect R^T_{\mmesh^-}\,\vect C_{\mmesh^-}\,\vect a_{\mmesh^-}}_{\vect s_{T^-}} \big)_{j+m}
				\end{split}
			\end{align*}
		}%
		\only<2>{
			\vskip -.1cm
			\begin{empheq}[box=\widefbox]{align*}
				\vect S_\bmesh &= \sum_{T \in \bmesh} \vect N^T_T\,\vect S_T\,\vect N_T, & \normalsize{\text{Global system:}} \\
				\vect s_\bmesh &= \sum_{T \in \bmesh} \vect N^T_T\,\vect s_T, & \normalsize{\textcolor{Red}{\vect S_\bmesh\,\vect \lambda_\bmesh = \vect s_\bmesh}}
			\end{empheq}
		}
		\end{overlayarea}
	\end{frame}
	
	\begin{frame}{ASC(1) System Assembly}\footnotesize
		\begin{empheq}[box=\widefbox]{align*}
			\vect S_\bmesh &= \sum_{T \in \bmesh} \vect N^T_T\,\vect S_T\,\vect N_T, & \normalsize{\text{Global system:}} \\
			\vect s_\bmesh &= \sum_{T \in \bmesh} \vect N^T_T\,\vect s_T, & \normalsize{\textcolor{Red}{\vect S_\bmesh\,\vect \lambda_\bmesh = \vect s_\bmesh}}
		\end{empheq}\normalsize
		\begin{itemize}
			\item \textbf{Theorem}: system matrix~$\vect S_\bmesh$ is sparse and SPD for ASC(0) and ASC(1)
			\item Hence efficient solvers and preconditioners are available (e.\,g. CG + Algebraic Multigrid)
			\item Once we obtain~$\vect \lambda_\bmesh$, we recover pressure and flux DOFs in each cell~$T \in \bmesh$ (this may be done in parallel) 
		\end{itemize}
	\end{frame}

	\section{Numerical Experiments}
	
	\subsection{ASC(0) $\rightarrow$ ASC(1): Motivation}
	
	\begin{frame}{ASC(1): Algebraic Robustness (1\,/\,2)}
		\begin{figure}
			\centering
			\caption{$w \coloneqq$ width of the left minimesh cells}
			\begin{subfigure}{.33\linewidth}
				\centering
				\includegraphicsw{skew1.png}
				\caption{$w = .1$}
			\end{subfigure}%
			\hfill
			\begin{subfigure}{.33\linewidth}
				\centering
				\includegraphicsw{skew01.png}
				\caption{$w = .01$}
			\end{subfigure}%
			\hfill
			\begin{subfigure}{.33\linewidth}
				\centering
				\includegraphicsw{skew001.png}
				\caption{$w = .001$}
			\end{subfigure}
		\end{figure}
		We solve the diffusion problem w/ $\vect K = k\,\vect I$, $k = 1$ on the left part and .1 on the right. Exact solution is piecewise linear
		\end{frame}
		
	\begin{frame}{ASC(1): Algebraic Robustness (2\,/\,2)}
		\begin{figure}
		\centering
		\caption{Condition Numbers of ASC(0)\,/\,ASC(1) Matrices} 
		\begin{subfigure}{.45\linewidth}
			\centering\tiny
			\begin{tabular}[1.2]{ | c | c | c | c |}
				\hline
				$w$ & $\kappa_{\text{ASC(0)}}$ & $\kappa_{\text{ASC(1)}}$ & $\widetilde\kappa_{\text{ASC(1)}}$\\
				\hline
				$10^{-1}$ & 41.0 & 1\,730       & 41.0  \\
				\hline
				$10^{-2}$ & 45.2 & 2\,817       & 45.1  \\
				\hline
				$10^{-3}$ & 48.3 & 16\,391      & 48.3   \\
				\hline
				$10^{-4}$ & 49.0 & 152\,325     & 49.0    \\
				\hline
				$10^{-5}$ & 49.1 & $1.5\times10^6$&49.1  \\
				\hline
			\end{tabular}%
%			\begin{tabular}[1.2]{ | c | c | c | }
%				\hline
%				$w$ & $\kappa_{\text{ASC(0)}}$ & $\kappa_{\text{ASC(1)}}$ \\
%				\hline
%				$10^{-1}$ & 41 & 1\,730 \\ 
%				\hline
%				$10^{-2}$ & 45 & 2\,817 \\
%				\hline
%				$10^{-3}$ & 48 & 16\,391 \\
%				\hline
%				$10^{-4}$ & 49 & 152\,325 \\
%				\hline
%				$10^{-5}$ & 49 & $1.5\times10^6$ \\
%				\hline
%			\end{tabular}
		\end{subfigure}%
		\hfill
		\begin{subfigure}{.55\linewidth}
			\centering
			\includegraphicsw{logplot.png}
		\end{subfigure}
		\end{figure}
		$\kappa_{\text{ASC(0)}}$ does not depend on $w$, and $\kappa_{\text{ASC(1)}}$ is proportional to $w^{-1}$. However, if we remove 3 smallest eig values (corresponding to 3 int MM faces), \textbf{we will have $\tilde{\vect\kappa}_{\text{ASC(1)}} \approx \vect\kappa_{\text{ASC(0)}}$}.
		Starting from some iteration CG behaves like extreme eig values are not present; that is, several small eig values is not a problem 
	\end{frame}

	\begin{frame}{$\lTwo$-Error}
		If the base mesh consists of triangles + we have no material interfaces, ASC($n$) boils down to Mixed-Hybrid Raviart\,--\,Thomas FEM:
		\begin{align*}
			\|\vect u - \vect u_h\|_{\LTwoSpace} &\le c\,h\,\|\vect u\|_{\HSpace{1}},\\
			\|p - p_h\|_{\LTwoSpace} &\le c \left( h\,\|p\|_{\HSpace{1}} + h^2\,\|p\|_{\HSpace{2}} \right).
		\end{align*}
		That is, we cannot expect ASC($n$) convergence to be better than linear. We define \textbf{discrete $\LTwo$-norm}
		\begin{equation*}
			\| v \|_{\lTwoSpace} \coloneqq \| P_h\,v \|_{\LTwoSpace} \le \| v \|_{\LTwoSpace},
		\end{equation*} 
		where $P_h \coloneqq$ $\LTwo$-projection operator on the space of piecewise constant functions on each cell~$T \in \bmesh$ (or on each~$\tau \in \mmesh$ if $T$ is a MMC)
	\end{frame}

	\begin{frame}{ASC(0) $\rightarrow$ ASC(1): Motivation}
		\begin{figure}
			\centering
			\begin{subfigure}{.45\linewidth}
				\centering
				\includegraphicsw{err2_asc0.png}
				\caption{$||p - p_h||_{\lTwoSpace} = 6.38 \times 10^{-2}$}
			\end{subfigure}%
			\hfill
			\begin{subfigure}{.45\linewidth}
				\centering
				\includegraphicsw{err3_asc0.png}
				\caption{$||p - p_h||_{\lTwoSpace} = 6.41 \times 10^{-2}$}
			\end{subfigure}
		\end{figure}
		Here $\vect K_i = \vect K_j$ and the exact soln is linear. ASC(0) produces errors due to const trace approximation, and ACS(1) recovers the exact soln
	\end{frame}
	
	\subsection{Piecewise $P_1$ \& $P_2$ Solutions}
	
	\begin{frame}{Piecewise Linear Solution (1\,/\,2)}
		We solve the diffusion problem on the sequence of square meshes w/ $\vect K = k\,\vect I$, $k = 1$ on the left part and .1 on the right. Exact solution is piecewise linear
		\begin{figure}
			\centering
			\begin{subfigure}{.45\linewidth}
				\centering
				\includegraphicsw{skew_ref.png}
				\caption{Exact soln, $p$}
			\end{subfigure}%
			\hfill
			\begin{subfigure}{.45\linewidth}
				\centering
				\includegraphicsw{skew_geometry_square.png}
				\caption{Materials}
			\end{subfigure}
		\end{figure}
	\end{frame}

	\begin{frame}{Piecewise Linear Solution (2\,/\,2)}
%		\only<1>{%
		\centering\small
		\begin{tabular}[1.2]{| c | c || c | c | c || c | c |}
			\hline
			\multirow{5}{*}{\rotatebox{90}{ASC(0)}} & $h$ & $\errlTwo{p}$ & $\rho_p$ & $\errInf{p}$ & $\errlTwo{u}$ & $\rho_u$ \\
			\cline{2-7}
			& $3.5\times10^{-1}$ & $7.3\times10^{-1}$ &     & 4.8               & $ 6.6\times10^{-1}$ &  \\
			\cline{2-7}
			& $8.8\times10^{-2}$ & $1.6\times10^{-1}$ & 1.1 & 1.2               & $ 3.5\times10^{-1}$ & 0.46 \\ 
			\cline{2-7}
			& $2.2\times10^{-2}$ & $3.7\times10^{-2}$ & 1.1 & $3.4\times10^{-1}$ & $ 1.3\times10^{-1}$ & 0.71 \\ 
			\cline{2-7}
			& $5.5\times10^{-3}$ & $8.9\times10^{-3}$ & 1.0 & $7.9\times10^{-2}$ & $ 4.1\times10^{-2}$ & 0.83 \\
			\hline
			\hline
			\multirow{5}{*}{\rotatebox{90}{ASC(1)}} & $h$ & $\errlTwo{p}$ & $\rho_p$ & $\errInf{p}$ & $\errlTwo{u}$ & $\rho_u$ \\
			\cline{2-7}
			& $3.5\times10^{-1}$ & $2.5\times10^{-2}$ & & $2.9\times10^{-1}$      & $ 4.6\times10^{-2}$ &   \\
			\cline{2-7}
			& $8.8\times10^{-2}$ & $1.9\times10^{-3}$ & 1.84 & $6.6\times10^{-2}$ & $ 2.0\times10^{-2}$ & 0.6 \\
			\cline{2-7}
			& $2.2\times10^{-2}$ & $1.6\times10^{-4}$ & 1.79 & $4.3\times10^{-2}$ & $ 5.5\times10^{-3}$ & 0.93 \\
			\cline{2-7}
			& $5.5\times10^{-3}$ & $1.3\times10^{-5}$ & 1.80 & $2.0\times10^{-2}$ & $ 1.3\times10^{-3}$ & 1.   \\
			\hline
		\end{tabular}
%		}%
%		\only<2>{%
%			\begin{figure}
%				\centering
%				\includegraphicsw[.8]{skew_conv_square.png}
%			\end{figure}
%		}
	\end{frame}

	\begin{frame}{Piecewise Quadratic Solutions w/ 2 Materials (1\,/\,3)}
		We solve the diffusion problem on Voronoi meshes w/ $\vect K = k\,\vect I$, $k = 1$ outside the circle and .001 inside. Exact solution is pw quadratic. We compare convergence of ASC(0) and ASC(1) 
		\begin{figure}
			\centering
			\begin{subfigure}{.45\linewidth}
				\centering
				\includegraphicsw{circle_ref_mesh.png}
				\caption{Exact soln, $p$}
			\end{subfigure}%
			\hfill
			\begin{subfigure}{.45\linewidth}
				\centering
				\includegraphicsw{circle_ref_slice.png}
				\caption{$p(x,\frac{1}{2})$}
			\end{subfigure}
		\end{figure}
	\end{frame}

	\begin{frame}{Piecewise Quadratic Solution w/ 2 Materials (2\,/\,3)}
%		\only<1>{%
			\centering\footnotesize
			\begin{tabular}[1.1]{| c | c || c | c | c ||c | c |}
				\hline
				\multirow{7}{*}{\rotatebox{90}{ASC(0)}} & $h$ & $\errlTwo{p}$ & ${\rho_p}$ & $\errInf{p}$ & $\errlTwo{u}$ & $\rho_u$ \\
				\cline{2-7}
				& $3.0\times10^{-1}$ & $2.4\times10^{-3}$ &     & $6.3\times10^{-1}$ & $6.9\times10^{-5}$ &  \\ %& $2.6\times10^{-4}$ \\
				\cline{2-7}
				& $1.5\times10^{-1}$ & $6.5\times10^{-4}$ & 2.0 & $7.0\times10^{-3}$ & $3.6\times10^{-5}$ & 0.9 \\ %&$1.6\times10^{-4} $ \\
				\cline{2-7}
				& $8.1\times10^{-2}$ & $2.6\times10^{-4}$ & 1.4 & $3.2\times10^{-3}$ & $3.4\times10^{-5}$ & 0.09 \\ %&$2.0\times10^{-3}$  \\
				\cline{2-7}
				& $4.2\times10^{-2}$ & $1.4\times10^{-4}$ & 0.9 & $2.3\times10^{-3}$ & $2.1\times10^{-5}$ & 0.73 \\ %&$1.0\times10^{-3}$  \\
				\cline{2-7}
				& $2.1\times10^{-2}$ & $3.7\times10^{-5}$ & 1.9 & $1.1\times10^{-3}$ & $1.1\times10^{-5}$ & 0.93 \\ %&$6.9\times10^{-4}$  \\
				\cline{2-7}
				& $1.0\times10^{-2}$ & $2.7\times10^{-5}$ & 0.4 & $8.6\times10^{-4}$ & $6.5\times10^{-6}$ & 0.75 \\ % &$1.8\times10^{-3}$  \\
				\hline
				\hline
				\multirow{7}{*}{\rotatebox{90}{ASC(1)}} & $h$ & $\errlTwo{p}$ & ${\rho_p}$ & $\errInf{p}$ & $\errlTwo{u}$ & $\rho_u$ \\
				\cline{2-7}
				& $3.0\times10^{-1}$ & $2.4\times10^{-3}$ & & $2.1\times10^{-3}$     &$1.2\times10^{-4}$  &  \\ %&$3.0\times10^{-4}$  \\
				\cline{2-7}
				& \hl{$1.5\times10^{-1}$} & $7.0\times10^{-4}$ & 1.9 & \hl{$1.3\times10^{-2}$} &$6.5\times10^{-5}$  & 0.88 \\ %&$8.8\times10^{-4}$  \\
				\cline{2-7}
				& $8.1\times10^{-2}$ & $2.3\times10^{-4}$ & 1.8 & $6.8\times10^{-4}$ &$3.0\times10^{-5}$  & 1.25 \\ %&$2.6\times10^{-4}$  \\
				\cline{2-7}
				& $4.2\times10^{-2}$ & $6.8\times10^{-5}$ & 1.8 & $3.2\times10^{-4}$ &$1.4\times10^{-5}$  & 1.16 \\ %&$3.2\times10^{-4}$  \\
				\cline{2-7}
				& $2.1\times10^{-2}$ & $2.0\times10^{-5}$ & 1.8 & $1.1\times10^{-4}$ &$5.0\times10^{-6}$  & 1.48 \\ %&$2.4\times10^{-4}$  \\
				\cline{2-7}
				& $1.0\times10^{-2}$ & $5.4\times10^{-6}$ & 1.9 & $3.3\times10^{-5}$ &$2.2\times10^{-6}$  & 1.18 \\ %&$1.2\times10^{-4}$  \\
				\hline
			\end{tabular}
%			\begin{tabular}[1.2]{| c | c || c | c || c |}
%				\hline
%				\multirow{6}{*}{\rotatebox{90}{ASC(0)\qquad\quad}} & $h$ & $\errlTwo{0}$ & $\mathcal p$ & $\errInf{0}$ \\
%				\cline{2-5}
%				& $3.0\times10^{-1}$ & $2.4\times10^{-3}$ & & $6.3\times10^{-1}$ \\ 
%				\cline{2-5}
%				& $1.5\times10^{-1}$ & $6.5\times10^{-4}$ & 2.0 & $7.0\times10^{-3}$ \\
%				\cline{2-5}
%				& $8.1\times10^{-2}$ & $2.6\times10^{-4}$ & 1.4 & $3.2\times10^{-3}$ \\
%				\cline{2-5}
%				& $4.2\times10^{-2}$ & $1.4\times10^{-4}$ & $9.4\times10^{-1}$ & $2.3\times10^{-3}$ \\
%				\cline{2-5}
%				& $2.1\times10^{-2}$ & $3.7\times10^{-5}$ & 1.9 & $1.1\times10^{-3}$ \\
%				\cline{2-5}
%				& $1.0\times10^{-2}$ & $2.7\times10^{-5}$ & $4.4\times10^{-1}$ & $8.6\times10^{-4}$ \\
%				\hline
%				\hline
%				\multirow{6}{*}{\rotatebox{90}{ASC(1)\qquad\quad}} & $h$ & $\errlTwo{1}$ & $\mathcal p$ & $\errInf{1}$ \\
%				\cline{2-5}
%				& $3.0\times10^{-1}$ & $2.4\times10^{-3}$ & & $2.1\times10^{-3}$ \\ 
%				\cline{2-5}
%				& \hl{$1.5\times10^{-1}$} & $7.0\times10^{-4}$ & 1.9 & \hl{$1.3\times10^{-2}$} \\
%				\cline{2-5}
%				& $8.1\times10^{-2}$ & $2.3\times10^{-4}$ & 1.8 & $6.8\times10^{-4}$ \\
%				\cline{2-5}
%				& $4.2\times10^{-2}$ & $6.8\times10^{-5}$ & 1.8 & $3.2\times10^{-4}$ \\
%				\cline{2-5}
%				& $2.1\times10^{-2}$ & $2.0\times10^{-5}$ & 1.8 & $1.1\times10^{-4}$ \\
%				\cline{2-5}
%				& $1.0\times10^{-2}$ & $5.4\times10^{-6}$ & 1.9 & $3.3\times10^{-5}$ \\
%				\hline
%			\end{tabular}
			\vskip .45cm
			We observe a jump of $\infty$-error of ASC(1) at \hl{$h = 1.5\times10^{-1}$}
%		}%
%		\only<2>{%
%			\begin{figure}
%				\centering
%				\includegraphicsw[.8]{circle_conv_voronoi.png}
%			\end{figure}
%			We observe a jump of $\infty$-error of ASC(1) at \hl{$h = 1.5\times10^{-1}$}
%		}
	\end{frame}

	\begin{frame}{Piecewise Quadratic Solution w/ 2 Materials (3\,/\,3)}
		\begin{figure}
			\centering
			\begin{subfigure}{.3\linewidth}
				\centering
				\includegraphicsw{circle_voronoi_2_mat.png}
				\caption{Materials}
			\end{subfigure}%
			\qquad
			\begin{subfigure}{.4\linewidth}
				\footnotesize{\hl{$h = 1.5\times10^{-1}$}: This example shows that ASC(1) $\infty$-norm may be sensitive to geometry errors. However, it does not affect $\lTwo$-convergence}		
			\end{subfigure}%
			\vfill
			\begin{subfigure}{.35\linewidth}
				\centering
				\includegraphicsw{circle_voronoi_2_asc0.png}
				\caption{ASC(0), $p_h$}
			\end{subfigure}%
			\qquad
			\begin{subfigure}{.35\linewidth}
				\centering
				\includegraphicsw{circle_voronoi_2_asc1.png}
				\caption{ASC(1), $p_h$}			
			\end{subfigure}%
		\end{figure}
	\end{frame}

	\begin{frame}{Piecewise Quadratic Solution w/ 3 Materials (1\,/\,2)}
		We solve the diffusion problem on triangular meshes w/ $\vect K = k\,\vect I$, $k = 1$ outside the ring and .001 inside. Exact solution is piecewise quadratic
		\begin{figure}
			\centering
			\begin{subfigure}{.45\linewidth}
				\centering
				\includegraphicsw{ring_ref_mesh.png}
				\caption{Exact soln, $p$}
			\end{subfigure}%
			\hfill
			\begin{subfigure}{.45\linewidth}
				\centering
				\includegraphicsw{ring_ref_slice.png}
				\caption{$p(x,\frac{1}{2})$}
			\end{subfigure}
		\end{figure}
	\end{frame}

	\begin{frame}{Piecewise Quadratic Solution w/ 3 Materials (2\,/\,2)}
		\Wider{\tiny
		\begin{tabular}[1.1]{| c | c || c | c || c |}
			\hline
			\multirow{7}{*}{\rotatebox{90}{ASC(0)}} & $h$ & $\errlTwo{p}$ & $\rho_p$ & $\errInf{p}$ \\
			\cline{2-5}
			& $3.0\times10^{-1}$ & 4.5 & & 17 \\
			\cline{2-5}
			& $2.5\times10^{-1}$ & 4.5 & & 17 \\
			\cline{2-5}
			& $1.3\times10^{-1}$ & 4.0 & & 17 \\
			\cline{2-5}
			& $8.3\times10^{-2}$ & 4.4 & & 17 \\
			\cline{2-5}
			& \hl{$6.7\times10^{-2}$} & $7.1\times10^{-1}$ & & 4.9 \\
			\cline{2-5}
			& $4.3\times10^{-2}$ & $4.5\times10^{-1}$ & 1.2 & 5.0 \\
			\hline
			\hline
			\multirow{7}{*}{\rotatebox{90}{ASC(1)}} & $h$ & $\errlTwo{p}$ & $\rho_p$ & $\errInf{p}$ \\
			\cline{2-5}
			& $3.0\times10^{-1}$ & $4.5\times10^{-1}$ & & 3.5 \\
			\cline{2-5}
			& $2.5\times10^{-1}$ & $2.6\times10^{-1}$ & 3 & 2.7 \\
			\cline{2-5}
			& $1.3\times10^{-1}$ & $9.2\times10^{-2}$ & 1.5 & $6.2\times10^{-1}$ \\
			\cline{2-5}
			& $8.3\times10^{-2}$ & $4.8\times10^{-2}$ & 1.6 & $8.3\times10^{-1}$ \\
			\cline{2-5}
			& $6.7\times10^{-2}$ & $2.8\times10^{-2}$ & 2.5 & $2.3\times10^{-1}$ \\
			\cline{2-5}
			& $4.3\times10^{-2}$ & $1.0\times10^{-2}$ & 2.3 & $6.3\times10^{-2}$ \\
			\hline
		\end{tabular}%
		\hfill
		\begin{tabular}[1.1]{| c | c || c | c || c |}
			\hline
			\multirow{7}{*}{\rotatebox{90}{Arithmetic homo.}} & $h$ & $\errlTwo{\text{p}}$ & $\rho_p$ & $\errInf{p}$ \\
			\cline{2-5}
			& $3.0\times10^{-1}$ & 4.9 & & 17 \\
			\cline{2-5}
			& $2.5\times10^{-1}$ & 5.0 & & 17 \\
			\cline{2-5}
			& $1.3\times10^{-1}$ & 4.9 & & 17 \\
			\cline{2-5}
			& $8.3\times10^{-2}$ & 4.7 & & 17 \\
			\cline{2-5}
			& {$6.7\times10^{-2}$} & 4.4 & & 16 \\
			\cline{2-5}
			& $4.3\times10^{-2}$ & $9.7\times10^{-1}$ & 3.5 & 5.7 \\
			\hline
			\hline
			\multirow{7}{*}{\rotatebox{90}{Harmonic homo.}} & $h$ & $\errlTwo{p}$ & $\rho_p$ & $\errInf{p}$ \\
			\cline{2-5}
			& $3.0\times10^{-1}$ & 2.3 & & 15 \\
			\cline{2-5}
			& $2.5\times10^{-1}$ & 1.7 & 1.6 & 16 \\
			\cline{2-5}
			& $1.3\times10^{-1}$ & $7.3\times10^{-1}$ & 1.2 & 12 \\
			\cline{2-5}
			& $8.3\times10^{-2}$ & $4.8\times10^{-1}$ & 1.0 & 12 \\
			\cline{2-5}
			& $6.7\times10^{-2}$ & $3.4\times10^{-1}$ & 1.6 & 9.4 \\
			\cline{2-5}
			& $4.3\times10^{-2}$ & $1.6\times10^{-1}$ & 1.7 & 8.2 \\
			\hline
		\end{tabular}
		\vskip .45cm
		\normalsize
		Before \hl{$h = 6.7\times10^{-2}$} we have cells\,/\,faces with 3 materials, and after this mesh level we have only 2 material MMCs
		}
%		\only<1>{%
%				\centering\footnotesize
%				\begin{tabular}[1.3]{| c | c || c | c || c |}
%					\hline
%					\multirow{7}{*}{\rotatebox{90}{ASC(0)}} & $h$ & $\errlTwo{0}$ & $\mathcal p$ & $\errInf{0}$ \\
%					\cline{2-5}
%					& $3.0\times10^{-1}$ & 4.5 & & 17 \\ 
%					\cline{2-5}
%					& $2.5\times10^{-1}$ & 4.5 & & 17 \\
%					\cline{2-5}
%					& $1.3\times10^{-1}$ & 4.0 & & 17 \\
%					\cline{2-5}
%					& $8.3\times10^{-2}$ & 4.4 & & 17 \\
%					\cline{2-5}
%					& \hl{$6.7\times10^{-2}$} & $7.1\times10^{-1}$ & & 4.9 \\
%					\cline{2-5}
%					& $4.3\times10^{-2}$ & $4.5\times10^{-1}$ & 1.2 & 5.0 \\
%					\hline
%					\hline
%					\multirow{7}{*}{\rotatebox{90}{ASC(1)}} & $h$ & $\errlTwo{0}$ & $\mathcal p$ & $\errInf{0}$ \\
%					\cline{2-5}
%					& $3.0\times10^{-1}$ & $4.5\times10^{-1}$ & & 3.5 \\ 
%					\cline{2-5}
%					& $2.5\times10^{-1}$ & $2.6\times10^{-1}$ & 3 & 2.7 \\
%					\cline{2-5}
%					& $1.3\times10^{-1}$ & $9.2\times10^{-2}$ & 1.5 & $6.2\times10^{-1}$ \\
%					\cline{2-5}
%					& $8.3\times10^{-2}$ & $4.8\times10^{-2}$ & 1.6 & $8.3\times10^{-1}$ \\
%					\cline{2-5}
%					& $6.7\times10^{-2}$ & $2.8\times10^{-2}$ & 2.5 & $2.3\times10^{-1}$ \\
%					\cline{2-5}
%					& $4.3\times10^{-2}$ & $1.0\times10^{-2}$ & 2.3 & $6.3\times10^{-2}$ \\
%					\hline
%				\end{tabular}
%		}%
%		\only<2>{%
%			\centering\footnotesize
%			\begin{tabular}[1.3]{c | c | c || c | c || c |}
%				\cline{2-6}
%				\multirow{14}{*}{\rotatebox{90}{Homogenization}} & \multirow{7}{*}{\rotatebox{90}{Arithmetic}} & $h$ & $\errlTwo{\text{AH}}$ & $\mathcal p$ & $\errInf{\text{AH}}$ \\
%				\cline{3-6}
%				& & $3.0\times10^{-1}$ & 4.9 & & 17 \\ 
%				\cline{3-6}
%				& & $2.5\times10^{-1}$ & 5.0 & & 17 \\
%				\cline{3-6}
%				& & $1.3\times10^{-1}$ & 4.9 & & 17 \\
%				\cline{3-6}
%				& & $8.3\times10^{-2}$ & 4.7 & & 17 \\
%				\cline{3-6}
%				& & \hl{$6.7\times10^{-2}$} & 4.4 & & 16 \\
%				\cline{3-6}
%				& & $4.3\times10^{-2}$ & $9.7\times10^{-1}$ & 3.5 & 5.7 \\
%				\hhline{~=====}
%				& \multirow{7}{*}{\rotatebox{90}{Harmonic}} & $h$ & $\errlTwo{\text{HH}}$ & $\mathcal p$ & $\errInf{\text{HH}}$ \\
%				\cline{3-6}
%				& & $3.0\times10^{-1}$ & 2.3 & & 15 \\ 
%				\cline{3-6}
%				& & $2.5\times10^{-1}$ & 1.7 & 1.6 & 16 \\
%				\cline{3-6}
%				& & $1.3\times10^{-1}$ & $7.3\times10^{-1}$ & 1.2 & 12 \\
%				\cline{3-6}
%				& & $8.3\times10^{-2}$ & $4.8\times10^{-1}$ & 1.0 & 12 \\
%				\cline{3-6}
%				& & $6.7\times10^{-2}$ & $3.4\times10^{-1}$ & 1.6 & 9.4 \\
%				\cline{3-6}
%				& & $4.3\times10^{-2}$ & $1.6\times10^{-1}$ & 1.7 & 8.2 \\
%				\cline{2-6}
%			\end{tabular}
%		}%
%		\only<3>{%
%			\begin{figure}
%				\centering
%				\includegraphicsw[.8]{ring_conv_triangular.png}
%			\end{figure}
%			Before \hl{$h = 6.7\times10^{-2}$} we have cells\,/\,faces with 3 materials, and after this mesh level we have only 2 material MMCs
%		}
	\end{frame}

	\subsection{Unsteady Problem}
	
	\begin{frame}{Unsteady Problem (1\,/\,2)}
		We solve the unsteady diffusion problem on Voronoi meshes w/ $\vect K = k\,\vect I$, $k = .002$ inside the ring, 10 in the inner disk, and 1 elsewhere. We set~$g_D = 1$ on the left bndry and 0 on the right; top and bottom bndries are insulated. Equilibrium state is $p \equiv 1$
		\begin{figure}
			\centering
			\begin{subfigure}{.4\linewidth}
				\centering
				\includegraphicsw{transient2/supermesh.png}
				\caption{Conforming (super)mesh}
			\end{subfigure}%
			\hfill
			\begin{subfigure}{.58\linewidth}
				\centering
				\includegraphicsw{transient2/ref_slices.png}
				\caption{Cuts~$p_*\big((x,0.5), t\big)$ of the ref soln}	
			\end{subfigure}
		\end{figure}
	\end{frame}

	\setbeamerfont{caption}{size=\tiny}

	\begin{frame}{Unsteady Problem (2\,/\,2)}\Wider{
		\begin{figure}
			\centering
			\begin{subfigure}{.25\linewidth}
				\centering
				\includegraphicsw{transient2/movie_frames/ref/frame_1s34.png}
				\caption{\tiny Reference}
			\end{subfigure}%
			\hfill
			\begin{subfigure}{.25\linewidth}
				\centering
				\includegraphicsw{transient2-1/movie_frames/arithmetic_homo/frame_1s43.png}
				\caption{\tiny Arithmetic homo.}	
			\end{subfigure}%
			\hfill
			\begin{subfigure}{.25\linewidth}
				\centering
				\includegraphicsw{transient2-1/movie_frames/harmonic_homo/frame_1s43.png}
				\caption{\tiny Harmonic homo.}	
			\end{subfigure}%
			\par
			\begin{subfigure}{.25\linewidth}
				\centering
				\includegraphicsw{transient2-1/movie_frames/asc0/frame_1s43.png}
				\caption{\tiny ASC(0)}
			\end{subfigure}%
			\hfill
			\begin{subfigure}{.25\linewidth}
				\centering
				\includegraphicsw{transient2-1/movie_frames/asc1/frame_1s43.png}
				\caption{\tiny ASC(1)}	
			\end{subfigure}%
			\hfill
			\begin{subfigure}{.25\linewidth}
				\centering
				\includegraphicsw{transient2-1/movie_frames/asc_diff/frame_1s43.png}
				\caption{\tiny ASC(0,\,1) difference}	
			\end{subfigure}%
		\end{figure}\centering
		Comparison of the discrete solutions~$p_h$, $h = 1.5 \times 10^{-1}$, $t = 1.25$ (\href{run:../img/transient2-1/movie.avi}{\ul{play}})
	}\end{frame}

	\begin{frame}{Summary}
		\textbf{Results}:
		\begin{itemize}
			\item ASC($n$) is able to efficiently handle unfitted material interfaces 
			\item 2\textsuperscript{nd} order $\lTwo$-convergence for ASC(1) 
			\item Effective condition number seems to be uniformly bounded w.r.t. an interface position
			\item The underline matrix is SPD and sparse; its pattern does not depend on mini\,meshes
			\item A.\,Zhiliakov et al. \href{https://www.researchgate.net/publication/330912268_A_higher_order_approximate_static_condensation_method_for_multi-material_diffusion_problems}{\ul{A higher order approximate static condensation method for multi-material diffusion problems}}, 2019 (JCP preprint)
		\end{itemize}
		\textbf{TODO List}:
		\begin{itemize}
			\item Anisotropic diffusion: homogenization is not applicable; what about ASC($n$)? 
		\end{itemize}
	\end{frame}

\end{document}


